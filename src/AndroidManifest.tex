%----
\subsection{AndroidManifest.xml}

В файле \textit{AndroidManifest.xml} описываются все компоненты Android приложения.
 
\lstset{
    language=xml,
    caption=Пример объявления компонента Activity,
    label=code:activity_xml,
    keywordstyle=\color{blue}\bf,
    commentstyle=\color{OliveGreen},
    stringstyle=\color{red},
    basicstyle=\scriptsize    
}
\lstinputlisting{files/activity.xml}
\begin{ESKDexplanation}
\item[где ] \textit{activity name} --- имя класса который реализует Activity в java;
\item \textit{activity lable} --- заголовок Activity;
\item \textit{activity windowSoftInputMode} --- указано что должна быть скрыта программная клавиатура;
\item \textit{activity launchMode} --- задано значение \textit{singleTask};
\item \textit{intent-filter action name} --- показывает что это главная Activity приложения, та с которой начнется запуск приложения;
\item \textit{intent-filter category name} --- показывает что Activity будет присутствовать в Laucnher-е Android, т.е. в списке всех приложений. 
\end{ESKDexplanation}

\subsubsection{Параметр configChanges}
	В этом параметре можно указать при каких изменениях мы не хотим чтобы Android пересоздавал Activity. Вместо этого будет вызван метод \textit{onConfigurationChanged()}, в котором мы обработаем смену конфигурации самостоятельно.
	 
	 
В пример ниже, при повороте экрана просто показывается его текущее состояние.
	\begin{center}
	android:configChanges=``orientation|ScreenSize''
	\end{center}
	
	\lstset{
	    language=java,
	    caption=Пример обработчика изменения конфигурации,
	    label=code:onConfigurationChanged_java,
	    keywordstyle=\color{blue}\bf,
	    commentstyle=\color{OliveGreen},
	    stringstyle=\color{red},
	    basicstyle=\scriptsize   
	}
	
	\lstinputlisting{files/onConfigurationChanged.java}
	
\warning{Использование параметра \textit{configChanges} не избавляет от необходимости корректно обрабатывать пересоздание Activity. \newline Оно оправдано только в редких, исключительных случаях!}