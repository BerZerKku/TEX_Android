%%This is a very basic article template.
%%There is just one section and two subsections.
\input{preambule}

% Заполнение граф Титульного листа и основной надписи
\ESKDcompany{ООО <<Прософт-Системы>>}
%\ESKDclassCode{}
\ESKDtitle{Android Manual}
\ESKDdocName{Отдел Энергосвязи}
%\ESKDsignature{Заметки OrCAD 16.6}
\ESKDgroup{\normalsize ООО <<Прософт-Системы>>}

% основная надпись
\ESKDauthor{\ESKDfontII Щеблыкин М.В.}
%\ESKDchecker{\ESKDfontII Макаров Е.Г.}
%\ESKDnormContr{\ESKDfontII Бунина О.Ю.}
%\ESKDapprovedBy{\ESKDfontII Чирков А.Г.}
\ESKDdate{2014/08/14}
%\ESKDcolumnI{АВАНТ К400 \\ \vspace{0.5cm} \ESKDfontIII{Инструкция по использованию протокола MODBUS}}

% переназначение в itemize символа первого уровня на жирную точку
\renewcommand\labelitemi{$\bullet$}

\begin{document}


\section{Android}

\subsection{Основные компоненты приложения. Жизненный цикл приложения}

Приложение само не решает когда завершиться, это делает ОС в тот момент, когда ей понадобятся ресурсы.

Приоритеты процессов:
\begin{itemize}
	\item процесс переднего плана (который содержит Activity с которыми взаимодействует пользователь в данный момент или обрабатывается \\ BroadcastReceiver, или процессы содержащие foreground методы);
	\item видимый процесс (с которыми пользователь в данный момент не взаимодействует, но их UI виден пользователю);
	\item сервисный процесс (в котором запущен хотя бы один сервис);
	\item процесс заднего плана (которые содержат Activity не видимые пользователю в данный момент);
	\item пустой процесс (которые не содержат ни одного компонента, например, данные оставшиеся от убитой Activity).
\end{itemize}

%----
\subsubsection{Activity}
\textbf{Activity} --- одно окно приложения.

\begin{itemize}
	\item может занимать весь экран или его часть;
	\item может быть запущена из других компонент приложения или из другого приложения;
	\item может возвращать результат.
\end{itemize}

\begin{figure}[H]
    \centering
    \includegraphics[width=0.5\textwidth]{activity_lifecycle}
    \caption{Жизненный цикл Activity}
    \label{fig:activity_lifecycle}
\end{figure}

На разных этапах существования Activity андроид вызывает различные методы, основные представлены на рисунке \ref{fig:activity_lifecycle}.

\begin{itemize}
	\item \textit{onCreate()} --- вызывается первым. В нем как правило создаются UI контролы.
	\item \textit{onStart()} --- вызывается в момент появления Activity на экране.
	\item \textit{onResume()} --- вызывается когда пользователь начинает взаимодействовать с Activity.
	\item \textit{onPause()} --- вызывается когда пользователь заканчивает работать с Activity. Если приложение осталось на экране и пользователь в дальнейшем возвращается обратно, идет вызов \textit{onResume()}.
	\item \textit{onStop()} --- вызывается когда Activity полностью уходит с экрана и перестает быть видимым.
	\item \textit{onRestart()} --- вызывается при возврате в Activity и далее \textit{onStart()}.
	\item \textit{onDestroy()} --- вызывается при сммене конфигурации или если пользователь закрыл Activity. Activity уничтожается.
\end{itemize}

Приложение может быть уничтожено после методов \textit{onPause()}, \textit{onStop()}. В этом случае следующие методы вызваны не будут, например \textit{onDestroy()}. За освобождение ресурсов при этом отвечает уже ОС.

\begin{figure}[H]
    \centering
    \includegraphics[width=0.8\textwidth]{diagram_backstack}
    \caption{Activity back stack}
    \label{fig:diagram_backstack}
\end{figure}

Как правило в приложении существует не одно Activity, а несколько. Когда пользователь вызывает из одной Activity другую, оно кладется сверху, а первая Activity уходит вниз (см. рисунок ). Таким образом образуется некий стек Activity. Когда пользователь нажимает кнопку \textit{back}, с этого стека верхняя Activity сносится, уничтожается и наверх поднимается та, что была под ней.

На рисунке \ref{fig:diagram_backstack} изображено поведение \textit{back stack} по умолчанию, которое можно изменять.

Для изменения поведения \textit{back stack} в Activity существует параметр \textit{Launch Modes}.
\begin{itemize}
	\item \textit{standart (default mode)} --- при каждом запуске Activity создается новый экземпляр Activiy и помещается на вершину \textit{back stack};
	\item \textit{singleTop} --- если в момент запуска экземпляр Activity уже находится на вершине стека, то новый экземпляр не создается, вместо этого вызывается метод \textit{onNewIntent()} у существующего экземпляра;
	\item \textit{singleTask} --- Activity запускается в отдельном \textit{Task}. Если экземпляр Activity уже существует, то у него вызывается метод \textit{onNewIntent()}, а все Activity лежащие в \textit{back stack} поверх этого экземпляра - уничтожаются.
	\item \textit{singleInstance} то же, что и \textit{singleTask}, но Activity является в своем таске единственной.
\end{itemize}

%----
\subsubsection{Service}
\textbf{Service} --- компонент для выполнения длительных фоновых задач.

\begin{itemize}
	\item не содержит графического интерфейса;
	\item может выполняться в том же процессе, что и само приложение, либо в отдельном;
	\item повышает значимость процесса с точки зрения Android.
\end{itemize}

Может быть использован, например, для проигрывания музыки в фоновом режиме.


%----
\subsubsection{BroadcastReceiver}
\textbf{BroadcastReceiver} --- приемник широковещательных сообщений.

\begin{itemize}
	\item получает сообщения от Android или других приложений;
	\item Примеры широковещательных сообщений:
	\begin{itemize}
		\item BOOT
		\item SCREEN\_OFF/ON
		\item CONNECTIVITY\_ACTION
	\end{itemize}
	\item должен обрабатывать сообщения быстро, длительные задачи можно делегировать сервису.
\end{itemize}

При слишком долгом выполнении, например, порядка 10 секунд, Android может убить этот процесс.


%----
\subsubsection{Content Provider}
\textbf{Content Provider} --- компонент для доступа к хранилищу данных.

\begin{itemize}
	\item используется для доступа к данным, хранимым Android, или другими приложениями;
	\item приложение может давать доступ к своим данным для других приложений, реализуя Content Provider;
	\item предоставляет данные в виде таблиц, реализцет методы query, insert, update, delete.
\end{itemize}


%----
\subsubsection{Intent}
\textbf{Content Provider} --- сущность для описания операции, которую требуется выполнить.

Используется для:
\begin{itemize}
	\item запуска Activity;
	\item запуска сервиса;
	\item отправки широковещательных сообщений;
	\item выполнения стандартных, преопределенных операций.
\end{itemize}


%----
\subsubsection{Компонент Application}
\textbf{Application} --- существует в любом Android приложении. 
Создается вместе с процессом и первым вызывается метод \textit{onCreate()}, в котором удобно инициализировать собственные компоненты или сторонние библиотеки, т.к. вызывается до старта любых Activity, сервисов и т.д.

Метод \textit{onConfigurationChanged()} вызывается когда меняется конфигурация ОС. Например, когда меняется ориентация экрана, подключается клавиатура и т.д.


%----
\subsubsection{Task}
\textbf{Task} --- множество Activity.  

Внутри лежит \textit{back\_stack}. Может уйти в \textit{background} и быть возвращенным в \textit{foreground}. 


%----
\subsubsection{AndroidManifest.xml}

В файле \textit{AndroidManifest.xml} описываются все компоненты Android приложения.
 
\lstset{
    language=xml,
    caption=Пример объявления компонента Activity,
    label=code:activity_xml,
    keywordstyle=\color{blue}\bf,
    commentstyle=\color{OliveGreen},
    stringstyle=\color{red},
    basicstyle=\footnotesize    
}
\lstinputlisting{activity.xml}

\begin{ESKDexplanation}
\item[где ] \textit{activity name} --- имя класса который реализует Activity в java;
\item \textit{activity lable} --- заголовок Activity;
\item \textit{activity windowSoftInputMode} --- указано что должна быть скрыта программная клавиатура;
\item \textit{activity launchMode} --- задано значение \textit{singleTask};
\item \textit{intent-filter action name} --- показывает что это главная Activity приложения, та с которой начнется запуск приложения;
\item \textit{intent-filter category name} --- показывает что Activity будет присутствовать в Laucnher-е Android, т.е. в списке всех приложений.
\end{ESKDexplanation}
\end{document}
